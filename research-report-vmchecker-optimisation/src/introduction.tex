% vim: set tw=78 sts=2 sw=2 ts=8 aw et ai:

VMchecker is an open sources project started by a group of teachers and
students from the Computer Science Department at the Politehnica University
of Bucharest. The goal was to automate testing of homework from the
Operating Systems and Compilers courses. Since then, it was been adopted by
several other courses, that had assessments ranging from high level
programming, to network simulations to kernel programming.

The architecture of VMchecker is very modular and uses several
technologies. The infrastructure is organized in three components:
\begin{itemize}
\item vmchecker-ui
\item vmchecker-storer
\item vmchecker-tester
\end{itemize}

The \emph{User Interface} is a web interfaces based on Google Web Toolkit,
a framework with a Java backend and a JavaScript frontend. The UI is used
by students to upload theit assessments, view the results of the automated
tests and view their final grade and feedback.

Assessments uploaded via the web UI, are stored in the repository, on the
\emph{Storer}. Each homework is archived in a Git repository. A student can
submit several revisions of the same assessment, each is tested, but only
the last one (current one) is displayed in the UI. The Storer manages a
queue of assessments that will be send out for checking. This component is
a collection of scripts written in Python or Bash and require 3rd party
modules like PyVix, an API for managing VMware virtual machines. Also on
the Storer, the tests for each homework are stored.

The submitted assessments from the queue on the Storer are passed to the
\emph{Tester}. The Tester has one or more virtual machines (VMware in the
current implementation) that can be started. After a fresh machine is
started, the generic homework tests are uploaded to the virtual machine,
then the submitted assessment. Both the tests and the submitted assessment
is compiled and the test is ran. The output of the tests is saved, along
with other system logs (like kernel logs in dmesg). The results are passed
back to the Stored, from where the UI can hand the results to the students.



