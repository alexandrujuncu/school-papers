% vim: set tw=78 sts=2 sw=2 ts=8 aw et ai:
The use of an automated testing system for homeworks or for contest
submissions is is commonly used in the academic world, specially in
computer science related fields. Such systems offer an efficient way of
collecting and storing homeworks and a quick, impartial, deterministic
and transparent way of grading them. An automated testing speeds up
considerably the grading time as compared to manual grading.

A secure implementation of a testing systems need some sort of containment
of the environment in which each test is ran. Different types of
virtualisation technologies can be used, from chroot jails to full
virtualisation, depending on the requirements of the tests.

An example of an automated homework testing system is a project called
\emph{VMchecker}. This open source tool is made up of several components,
among which a virtualisation infrastructure based on VMware Workstation.
VMchecker has been successfully deployed in academic environments and is
currently serving over 500 students each semester.

The following paper is focused around optimising VMchecker in order to
better prioritise homework submissions to achieve faster grading times and
scaling the infrastructure to support parallel homework evaluations.

In order to do this, statistics about the current implementation of
VMchecker has been gathered Following the analysis of the data, a new
algorithm is being proposed for implementation.
