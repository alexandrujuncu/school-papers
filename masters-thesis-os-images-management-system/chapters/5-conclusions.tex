% vim: set tw=78 sts=2 sw=2 ts=8 aw et ai:

\chapter{Conclustions}\label{ch:conclusions}
\bigskip


The previous chapters discussed many technologies and
protocols that each provided needed features. But an efficient
management procedure would make use of a combination of all of them.
The following are some recommended procedures that resulted from the
research and experimenting done.

First of all, the old freezing system should be phased out. Because of
its lack of current development it has become inefficient for the
present needs. The "clean environment" features could be replaced by
running Live Linux distributions and using virtual machines.

Image multiplication actions that take place at least once per semester
or at every course beginning should be heavily reduced. Every
workstation should be imaged with a standard operating system that could
self update itself.

The standard operating system should be ready do use virtualization
software, like \ac{KVM} to run virtual systems if needed. It should also
have a pool of Live distribution iso files available for Grub2. The iso
as well as the virtual machines should be kept up to date by
synchronizing over the network with protocols such as \ac{NFS} or rsync.

In case a machine is malfunctioning, it could be remotely accessed using
vPro AMT and fixed remotely.

As we saw, using the approaches presented, projects can benefit from new
features, like VMchecker. Having a cluster or workstations that could be
used non stop and having a easy to use administration system to
provision resources and push new operating system images towards them in
order to accomplish a desired task, could greatly increase the efficiency
of hardware utilization and lowering costs processing power.


\section{Future work}

Being the newest technology presented, Intel's vPro AMT is a framework
where new research could be done. Because it is new, there is a lack of
software that exploits its features. But because it is a feature-full
framework many new ideas could be built on top.

As discussed previously, a scheduling system that would push images
automatically to workstations following timetables in a database could
reduce some manual tasks being done by an administrator that wants to
provision a pool of systems. Such a software project doesn't exist at
the time of writing this thesis and could be a starting point for a new
project.

The idea of a self-updating system has been discussed in this paper, in
order to make workstations automatically use the latest versions of
system images (either Live systems or virtual machines). Going further,
a self-repairing system could take one more set towards eliminating jobs
that an administrator would do. In a distributed self-repairing system,
one node could verify is another is malfunctioning and use its own
system to mirror the other node.

As the VMchecker case study has shown, there are some types of projects
that would benefit from a model where services could benefit on-the-fly
from workstations that aren't used for something else. Searching for
other case studies could prove to be important research.

