%*******************************************************
% Abstract
%*******************************************************
%\renewcommand{\abstractname}{Abstract}
\pdfbookmark[1]{Abstract}{Abstract}
\begingroup
\let\clearpage\relax
\let\cleardoublepage\relax
\let\cleardoublepage\relax

\chapter*{Abstract}

In recent years, many technologies have either evolved to have been
newly developed in hardware and software, in operating systems and
networking, to bring solutions like cloud computing front and center.


Virtualisation, distributed systems, redundancy, big data and high speed
are key words for the world of cloud computing. The technologies that
have been built around these concepts have shaped the modern Internet
and the domain of IT. A large number of businesses rose up during a short
period because of these technologies and idea. IT companies, ISPs and
non IT enterprises invested in them because of the benefits they were
getting: scalability to a large number of clients, cost effective ways
of managing their IT infrastructure and frameworks on which to build new
solutions.

Technical universities, especially Computer Science and Engineering
departments are environments where the IT infrastructure is not only
important, but essential to keeping the University operational. With its
many computer laboratories and server rooms, such environment has a
processing power  equal to a small or medium data center.

This thesis proposes a series of lessons that can be learned from the
world of enterprise cloud computing that can be implemented in university
environments in order to provide efficient use of equipment and
processing power. It will cover technologies for virtualisation like KVM
and OpenVZ/LXC, hardware features like Intel vPro, as well as case
studies of how available technologies can be used to fix current
problems.


\endgroup

\vfill
