

\chapter{The Imaging Server}\label{ch:server}

Like the client, the server is a Python script called  \emph{server.py}.
The main purpose of the server is to store the images created on the
workstations in the network. For this, it used a hierarchy in the file
system that stores images in binary files, with meta information in
other files.

All the images are stored in the \emph{images/} directory. For each image
there is an unique ID with a corresponding directory with the ID as the
name. In each ID-directory the is a \emph{info} file that contains the
description and a directory for each disk/partition in the image. Each
partition or disk directory has one or more directories that contain a
version of that image-disk.

\begin{lstlisting}[caption= Images directory hierarchy on the Image Server]
images/
|-- 1
|   |-- info
|   |-- sda
|   |   |-- 1
|   |   |-- 2
|   |   `-- 3
|   `-- sdb
|       `-- 1
|-- 10
|   `-- info
|-- 11
|   `-- info
|-- 12
|   `-- info
|-- 2
|   |-- info
|   `-- sda
|       |-- 1
|       |-- 2
|       |-- 3
|       |-- 4
|       `-- 5
|-- 3
|   |-- info
|   |-- sda1
|   |   `-- 1
|   `-- sda5
|       `-- 1
`-- 5
    `-- info

\end{lstlisting}

The Image Server awaits connections from clients. Because only one image
can be transfered at a time, only one client can connect at a time.

From the clients, the following requests can be received:
\begin{itemize}
\item LIST
\item CREATE
\item UPDATE
\item DELETE
\item REQUEST
\end{itemize}

The LIST actions is a request to send to the client information about the
images in the server's image database. The request can either be sent with
a value of 0, in which case the server will send information on all the
images, or with a value that is an ID of an image, in which case, only
information about the specified image is sent back to the client.

The CREATE, UPDATE and DELETE actions involve modifying  the image
database. CREATE will make a new image with a new ID and a directory in the
\emph{images/} directory, with a base version 1, while UPDATE will only
create a new version of a partition or disk in an existing image. DELETE
will erase an image.

In the case of CREATE and UPDATE, the client will ask the server to start a
udp-receiver process. The client has a udp-sender process started. When the
client gives the command, the server will receive the data and store the
sent partition or disk image in the local database.

When a REQUEST command is sent, the server will provide via a udp-sender
instance the content of a partition or disk to the upd-sender clients in
the network.
