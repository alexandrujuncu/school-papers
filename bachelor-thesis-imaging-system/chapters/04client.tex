

\chapter{The Imaging Client}\label{ch:client}

The client is represented by the \emph{client.py} Python script. The
requirements for running it are:
\begin{itemize}
\item the Python interpreter installed (multi platform)
\item the UDP Cast udp-sender and udp-receiver installed
\end{itemize}

There are several possibilities or running an Imaging Client:
\begin{enumerate}
\item from a locally installed operating system
\item from a separate partition on the local system
\item from a network boot image
\end{enumerate}

The first one is not recommended because if you want to copy the image of
the running operating system, it's contents might change during the
transfer and will invalidate the image.

A network boot should be the best option, because a server is already in
place and could be setup to service a \ac{PXE} image that contains the script.

The Image Client has a command line interface that accepts the commands
supported.

\begin{lstlisting}[caption= Image Client CLI]
**************************************
*Welcome to the Imaging System Client*
**************************************

Available commands:
l: list images on servers
c: create image
u: update image
d: delete image
r: request image
>>
\end{lstlisting}

Each command sends a request to the Image Server to execute an action.


\section{Image listing}

The first command is a request for the server to list the stored images in
the database. The client sends a "LIST" request and waits for the answer.

If the server receives a "LIST" request, it explores the \emph{images}
directory and it's contents. Each directory represents the ID of a system
image. In each ID-directory the is a \emph{info} file that contains the
description and a directory for each disk/partition in the image. Each
partition or disk directory has one or more directories that contain a
version of that image-disk.

Based on the information in the \emph{file} and the structure of the
directory hierarchy, the server sends a reply to the clients with the
needed information. The Client then displays the information to the user.

\section{Image Creation}

The creation of an image involves work both on the client and on the
server. 

When an image in created, the Image Client must be ran on the workstation
that contains the operating system(s) that will be imaged on the other
hosts. The client will scan the workstation for disks and partition.

A command line wizard will be started and the user will have to enter a
description for the image.

\begin{lstlisting}[caption=Image creation wizard on Image Client]
Available commands:
l: list images on servers
c: create image
u: update image
d: delete image
r: request image
>>c
 Creating image...
Available disks:
* sda
   - sda5
   - sda2
   - sda1
Fill in description for the image (press CTRL-D to finish reading input)
Image for room EG306-USO (Ubuntu 10.4 Customised)
Start transfer to server (it might take a long time)?[y/n]

\end{lstlisting}

The search for the devices is done using the \emph{/dev} system directory
in Linux.


\begin{lstlisting}[float, caption=Code snip of device listing]
def create_image():
	print "Creating image..."

	# Searching local disks and partitions 
	disks =  filter(re.compile("[sh]d.$").search, os.listdir("/dev"))
	print "Available disks:"
	for disk in disks:
		print "*", disk
		partitions =  filter(re.compile(disk + ".+$").search,
os.listdir("/dev"))
		for partition in partitions:
			print "   -", partition
\end{lstlisting}

This information is sent to the server that prepares a new ID for the
image. It will be considered version 1 of this image. The description is
stored in the \emph{info} file on the server in the corresponding directory.

After the devices to be copied are chosen, the client signals the server to
start an udp-receiver process to listen for the data. After the server has
started the process, the client start a local udp-sender with the input
being the selected partition(s).

At this point, on other workstations, Image Clients in request image mode
or normal UDP Cast clients can be started to also receive the image. This
saves time by both imaging all the hosts and archiving the image on the
server.

When all the receivers are connected, the transfer is started.

\section{Image Update}

Once an image for a workstation is on the server in an initial version, new
versions of a disk or a partition can be uploaded to the server. The
operation is similar to the creation, only the server does not create a new
ID, but only listens to new data images to be stored in the same directory
hierarchy.

\section{Image Deletion}

To free up space on the server old images can be deleted. The user can either
delete an entire image entry or just a version of a disk or a partition.

\section{Image Request}

The image request operation is similar to what UDP Cast does as a receiver.
A request for an image ID is send to the server, that starts a udp-sender
process with the selected version of a disk or partition.

All the workstations that need to be imaged, need to run the Image Client
in Image Request mode or run the UDP Cast receiver.
