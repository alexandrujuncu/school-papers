

\chapter{Conclusions and Future Development}\label{ch:conclusions}


The implementation of the project offers new features to the UDP Cast
framework, making is better suited for environments such as School and
University Computer Laboratories.

The main benefit of using the proposed architecture for system imaging is
the time saved in various situations. The costs of implementing such a
system is very small so it is appealing to be used.

The current version of the software implements the idea of a centralised
system in a very basic way. Some of the current shortcomings, that can be
mended in future development include:

\begin{itemize}
\item The Server can only serve one client at a time. This is because of
the way UDP Cast works, because it was designed to have one transfer at a
time. A modified implementation could make the server have udp-senders on
multiple multicast addresses and by keeping track of the source address,
more then one transfer could be made. This could lead to some
synchronisation problems on the image database and since the resources in
the archive are large files, locking could be an issue.
\item The client to server requests are done without authentication. Anyone
with a client can connect to the sever and add, remove and modify images.
This could be secured by creating a database of users (with different
privilege levels) and make the server ask for credentials before receiving
any requests from clients. Furthermore, the use of SSL/TLS could make the
control traffic immune to traffic sniffing.
\item The Client only offers a \ac{CLI} for sending requests to the sever and
doing any actions. A \ac{GUI} could provide a more user friendly
environment for users.
\item New versions of an image means copying new files to the server. These
files are very big in size (10-100GB) and could easily occupy the server's
disk space. A differential versioning system could reduce the file sizes by
making version 1 a base version and only storing the differences between
the base version and the new version on the server's disk.
\end{itemize}



